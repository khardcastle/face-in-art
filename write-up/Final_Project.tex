\documentclass{article}
\usepackage[final]{nips_2017}
\usepackage[utf8]{inputenc} % allow utf-8 input
\usepackage[T1]{fontenc}    % use 8-bit T1 fonts
\usepackage{hyperref}       % hyperlinks
\usepackage{url}            % simple URL typesetting
\usepackage{booktabs}       % professional-quality tables
\usepackage{amsfonts}       % blackboard math symbols
\usepackage{nicefrac}       % compact symbols for 1/2, etc.
\usepackage{microtype}      % microtypography
\usepackage{graphicx}
\title{Make yourself into a work of art}

\author{
  Kiah Hardcastle* and Julie Makelberge** \\
  Neuroscience PhD Program*, Graduate School of Business** \\
  Stanford University\\
  \texttt{khardcas@stanford.edu*, juliemkb@stanford.edu**} \\
  %% examples of more authors
  %% \And
  %% Coauthor \\
  %% Affiliation \\
  %% Address \\
  %% \texttt{email} \\
  %% \AND
  %% Coauthor \\
  %% Affiliation \\
  %% Address \\
  %% \texttt{email} \\
  %% \And
  %% Coauthor \\
  %% Affiliation \\
  %% Address \\
  %% \texttt{email} \\
  %% \And
  %% Coauthor \\
  %% Affiliation \\
  %% Address \\
  %% \texttt{email} \\
}

\begin{document}
% \nipsfinalcopy is no longer used

\begin{center}
\includegraphics[width=3cm, height=0.7cm]{CS230}
\end{center}

\maketitle

\begin{abstract}
In this paper, we outline the implementation of a neural style transfer framework that will integrate a person's face in an artistic painting. For example, given a headshot and a picture of the Mona Lisa, this framework would inpaint the headshot face into the face of the Mona Lisa, while retaining the artistic style of the Mona Lisa. There are a number of existing applications that have focussed face swapping in recent years, however they have generally focussed on doing so for photograph images. Furthermore, the few methodologies that have used neural style transfer to adjusting such photos to an artistic style, have constrained their solution to well-aligned images that contain only a face (i.e., cropped images), without taking into account the pose of the face. Our approach is novel in that we aim to retaining the style and pose of the original work of art's face, regardless of the pose of the supplied image, and will reintroduce the face in the work of art as a whole. 
\end{abstract}

\section{Introduction}  
Neural style transfer is the technique of using deep neural networks to transfer the style of a given reference image to the content of another. Traditionally, such problems have been adressed with classic image processing techniques such as histogram matching \cite{neumann2005color}. However, in 2015 Gatys et all. \cite{gatys2015neural} introduced a novel technique that leverages the power of Convolutional Neural Networks to emulate famous painting styles in natural images. In their seminal paper, they proposed to use the encoding from different layers of a pre-trained CNN, to capture the style elements and content elements of images. They then used an interative approach to optimise an image with the objective of minimizing the distance between the style elements of the image and the reference work of art, while keeping changes to the content as small as possible. 
\par
Their work has inspired a number of new neural transfer algorithms, ranging from general models such as the work of Li and Wand \cite{li2016combining} that combined generative Markov random field models with deep convolutional neural networks, to highly specialised domain-specific models such as Jiang anf Fu's Fashion style generator \cite{jiang2017fashion}. While general models have shown great potential in a large number of applications, they have generally introduced visual artifacts which are especially striking in faces, given human's sensitivity for facial irregularities. The work of Selim et all. \cite{selim2016painting} introduced the first approach for single-example based head portrait painting not constrained to a specific style. Their approach was successful by introducing additional constraints that exploit human facial geometry through the notion of gains maps. However, their methodology focused on front-facing cropped facial images and does not yet generalise to different facial positions or introduction within the wider context of a full painting. 
\par
In this work, we aim to expand on the work of Selim et all by introducing a framework that combines different Neural Transfer techniques to tackle the challenge of introducing faces of slightly varing positions in a work of art. Our goal is to create a resulting image that looks similar to our reference work of art, but with the exception of having a different person in mind at the time of painting. This technique would be valuable in the world of generative art as it alows different and personalised renderings of the same types of art work. 

\section{Related work}
You should find existing papers, group them into categories based on their approaches,
and discuss their strengths and weaknesses, as well as how they are similar to and differ
from your work. In your opinion, which approaches were clever/good? What is the stateof-the-art?
Do most people perform the task by hand? You should aim to have at least
5 references in the related work. Include previous attempts by others at your problem,
previous technical methods, or previous learning algorithms. Google Scholar is very useful
for this: https://scholar.google.com/ (you can click “cite” and it generates MLA, APA,
BibTeX, etc.)

\section{Dataset and Features}
Describe your dataset: how many training/validation/test examples do you have? Is there
any preprocessing you did? What about normalization or data augmentation? What is the
resolution of your images? How is your time-series data discretized? Include a citation on
where you obtained your dataset from. Depending on available space, show some examples
from your dataset. You should also talk about the features you used. If you extracted
features using Fourier transforms, word2vec, PCA,
ICA, etc. make sure to talk about it. Try to include examples of your data in the report
(e.g. include an image, show a waveform, etc.).



\section{ Methods }
Describe your learning algorithms, proposed algorithm(s), or theoretical proof(s). Make
sure to include relevant mathematical notation. For example, you can include the loss function you are using. It is okay to use formulas from the lectures (online or in-class). For each algorithm, give a short description 
of how it works. Again, we are looking for your understanding of how these deep
learning algorithms work. Although the teaching staff probably know the algorithms, future
readers may not (reports will be posted on the class website). Additionally, if you are
using a niche or cutting-edge algorithm (anything else not covered in the class), you may want to explain your algorithm using 1/2
paragraphs. Note: Theory/algorithms projects may have an appendix showing extended
proofs (see Appendix section below).

\section{Experiments/Results/Discussion}
You should also give details about what (hyper)parameters you chose (e.g. why did you
use X learning rate for gradient descent, what was your mini-batch size and why) and how
you chose them. What your primary metrics are: accuracy, precision,
AUC, etc. Provide equations for the metrics if necessary. For results, you want to have a
mixture of tables and plots. If you are solving a classification problem, you should include a
confusion matrix or AUC/AUPRC curves. Include performance metrics such as precision,
recall, and accuracy. For regression problems, state the average error. You should have
both quantitative and qualitative results. To reiterate, you must have both quantitative
and qualitative results! If it applies: include visualizations of results, heatmaps,
examples of where your algorithm failed and a discussion of why certain algorithms failed
or succeeded. In addition, explain whether you think you have overfit to your training set
and what, if anything, you did to mitigate that. Make sure to discuss the figures/tables in
your main text throughout this section. Your plots should include legends, axis labels, and
have font sizes that are legible when printed.

\section{Conclusion/Future Work }
Summarize your report and reiterate key points. Which algorithms were the highestperforming?
Why do you think that some algorithms worked better than others? For
future work, if you had more time, more team members, or more computational resources,
what would you explore?

\section{Contributions}
The contributions section is not included in the 5 page limit. This section should describe
what each team member worked on and contributed to the project.

\section*{References}
This section should include citations for: (1) Any papers mentioned in the related work
section. (2) Papers describing algorithms that you used which were not covered in class.
(3) Code or libraries you downloaded and used. This includes libraries such as scikit-learn, Tensorflow, Pytorch, Keras etc. Acceptable formats include: MLA, APA, IEEE. If you
do not use one of these formats, each reference entry must include the following (preferably
in this order): author(s), title, conference/journal, publisher, year. If you are using TeX,
you can use any bibliography format which includes the items mentioned above. We are excluding
the references section from the page limit to encourage students to perform a thorough
literature review/related work section without being space-penalized if they include more
references. Any choice of citation style is acceptable
as long as you are consistent. 

\medskip
\small

\bibliography{References}
\bibliographystyle{plain}

\end{document}